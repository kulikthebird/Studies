%!TEX TS-program = xelatex


%%%%%%%%%%%
% General %
%%%%%%%%%%%
\documentclass[A4,9pt]{article}
\NeedsTeXFormat{LaTeX2e}
\ProcessOptions\relax
\RequirePackage[left=1.5cm,top=1.5cm,right=1.5cm,bottom=1cm,nohead,nofoot]{geometry}


%%%%%%%%%%
% Colors %
%%%%%%%%%%
\RequirePackage{xcolor}
\definecolor{white}{RGB}{255,255,255}
\definecolor{darkgray}{HTML}{333333}
\definecolor{gray}{HTML}{4D4D4D}
\definecolor{lightgray}{HTML}{999999}
\definecolor{green}{HTML}{3F913F}
\definecolor{orange}{HTML}{FDA333}
\definecolor{purple}{HTML}{3C30B1}
\definecolor{red}{HTML}{FB4485}
\definecolor{pblue}{HTML}{0395DE}
\colorlet{fillheader}{white}
\colorlet{header}{gray}
\colorlet{textcolor}{gray}
\colorlet{headercolor}{gray}


%%%%%%%%%
% Fonts %
%%%%%%%%%
\RequirePackage[quiet]{fontspec}
\RequirePackage[math-style=TeX]{unicode-math}
\newfontfamily\bodyfont
[BoldFont=texgyreheros-bold.otf,
ItalicFont=texgyreheros-italic.otf,
BoldItalicFont=texgyreheros-bolditalic.otf]
{texgyreheros-regular.otf}
\newfontfamily\thinfont[]{Lato-Light.ttf}
\newfontfamily\headingfont[]{texgyreheros-bold.otf}
\defaultfontfeatures{Mapping=tex-text}
\setmainfont
[
  Mapping=tex-text, Color=textcolor,
  BoldFont=texgyreheros-bold.otf,
  ItalicFont=texgyreheros-italic.otf,
  BoldItalicFont=texgyreheros-bolditalic.otf
]
{texgyreheros-regular.otf}
\setmathfont{texgyreheros-regular.otf}


%%%%%%%%%%%%%
% Structure %
%%%%%%%%%%%%%
\RequirePackage{parskip}
\newcounter{colorCounter}
\def\@sectioncolor#1#2#3{%
  {%
    \color{%
      \ifcase\value{colorCounter}%
        pblue\or%
        pblue\or%
        pblue\or%
        pblue\or%
        pblue\else%
        pblue\fi%
    } #1#2#3%
  }%
  \stepcounter{colorCounter}%
}
\renewcommand{\section}[1]{
  \par\vspace{\parskip}
  {
    \LARGE\headingfont\color{headercolor}%
    \@sectioncolor #1%
  }
  \par\vspace{-1.2mm}
}
\renewcommand{\subsection}[2]{
  \par\vspace{.5\parskip}%
  \Large\headingfont\color{headercolor} #2%
  \par\vspace{.25\parskip}%
}
\pagestyle{empty}


%%%%%%%%%%%%%%%%%%%%
% List environment %
%%%%%%%%%%%%%%%%%%%%
\setlength{\tabcolsep}{0pt}
\newenvironment{entrylist}{%
  \begin{tabular*}{\textwidth}{@{\extracolsep{\fill}}ll}
}{%
  \end{tabular*}
}
\renewcommand{\bfseries}{\headingfont\color{headercolor}}
\newcommand{\entry}[4]{
{
  \footnotesize #1}&\parbox[t]{16.4cm}{
    \textbf{#2} \\
    {\footnotesize\addfontfeature{Color=pblue} #3} \\
    {\footnotesize #4}\vspace{\parsep}
  } \\
}


%%%%%%%%%%%%%%%%%%%%
% Imports          %
%%%%%%%%%%%%%%%%%%%%
\usepackage{afterpage}
\usepackage{hyperref}
\usepackage{color}
\usepackage{xcolor}
\usepackage{graphicx}
\RequirePackage{tikz}
\hypersetup {
    pdftitle={},
    pdfauthor={},
    pdfsubject={},
    pdfkeywords={},
    colorlinks=false,
    allbordercolors=white
}


%%%%%%%%%%%%%%%%%%%%
% Document         %
%%%%%%%%%%%%%%%%%%%%
\begin{document}
\begin{minipage}[c]{0cm}
  \-\
\end{minipage}
\begin{minipage}[c]{0.2\textwidth}
  \begin{tikzpicture}
    \clip (0,0) circle (1.75cm);
    \node at (0.17, 0.0) {\includegraphics[width = 4.3cm]{photo}};    % (x, y) and width=zoom
  \end{tikzpicture}
  \hfill\vline\hfill
\end{minipage}
\begin{minipage}[c]{0.35\textwidth}
  {
    \fontsize{30pt}{62pt}\color{header}
    {\thinfont Tomasz}  {\bodyfont Kulik}
  }
  \textbf{phone:} +48 660 718 720 \\
  \href{mailto:}{\textbf{tomek.kulik2@}gmail.com} \\
  \href{https://linkedin.com/in/tomkulik}{\textbf{linkedin.com}/in/tomkulik} \\
  \href{https://github.com/kulikthebird}{\textbf{github.com}/kulikthebird} \\
\end{minipage}

\fcolorbox{white}{gray}{\parbox{\dimexpr\textwidth-2\fboxsep-2\fboxrule}{.....}}


\section{About me}
\small
{
  % Info about systemd, linux etc. in nokia related part
  % More details about the work done in nokia (automation test / developer)
  % more info about code review and updating the process
  % Add some general information about what I am doing here in lxft
  % I have started as a Working Student in Nokia by the end of a 2nd year of studies. I was part of Q&A team that worked on
  % tests for base station. I had a chance to work on both a very large projects with thousands lines of code
  % and smaller applications written from scratch. I'm using SOLID principles, well known design patterns and unit testing in my daily basis.
  % My studies were focused mainly on theory of Computer Science, however there were a lot of practical exercises. I was using a number of
  % programming languages during labs. I learned a lot of algorithms from different fields such as networking, distributed systems,
  % coding and compression of data, parsing and compilers' architecture, machine learning, optimization etc.
  Some text here TODO.
}


\section{Experience}
\begin{entrylist}
  \entry
  {06/20 - present}
  {Software Developer}
  {Luxoft (Metaswitch), Wrocław, Poland}
  {
    \begin{itemize}
      \setlength\itemsep{0.2em}
      \item Development of 5G core software.
      \item Implementing new features and maintaining legacy code
      \item Faults analysis
      \item Python3 and C++14 code reviewer, member of a Scrum team
      \item Programming: C++14, Python3, TTCN-3, VBA
    \end{itemize}
  }
  \entry
  {08/18 - 06/20}
  {Engineer Software Developer}
  {Nokia, Wrocław, Poland}
  {
    After getting more experience in the project and programming skills I decided
    to focus more on the software development. Besides Python3, I started
    writing code for the telecommunication embedded system in C++14. I was
    also responsible for writing tests in TTCN-3 language. There was
    also other projects written mainly in Python in which I took part.
    \begin{itemize}
      \setlength\itemsep{0.2em}
      \item Development of LTE and IoT software in System Module and Radio Module
      \item Implementing new features and maintaining legacy code
      \item Faults analysis
      \item Python3 and C++14 code reviewer, member of a Scrum team
      \item Programming: C++14, Python3, TTCN-3, VBA
    \end{itemize}
  }
  \entry
  {06/15 - 08/18}
  {Software Test Engineer}
  {Nokia, Wrocław, Poland}
  {
    I have started as a Working Student in Nokia by the end of a 2nd year of studies.
    I was part of Q team that worked on tests for base station. After few months I
    switched to the full time job and became a Test Engineer. To get a wider view of
    the project I took part in the review process as a code reviewer. The job gave
    me a chance of working close to the 'bear metal' and simultaneously to learn
    good practices in programming. 
    \begin{itemize}
      \setlength\itemsep{0.2em}
      \item Functional Testing, Integration Testing
      \item Working on a test environment for LTE software (Base station)
      \item Writing automated tests, preparing documentation of test plans
      \item Python3 code reviewer, member of a Scrum team
      \item Programming: Python3, RobotFramework
    \end{itemize}
  }
\end{entrylist}


\section{Education}
\begin{entrylist}
  \entry
  {2017 - 2019}
  {Master's Degree in Computer Science (Algorithmics)}
  {Wrocław University of Science and Technology}
  {
    \emph{Faculty:} Fundamental Problems of Technology \\
    \emph{Thesis: ,,Filtering algorithms in constraints programming''.} \\
    \emph{Description and implementation of algorithms for domains reduction
      (constraints propagation) within IBM CPLEX environment. \textbf{C++11 / Optimization} }
  }
  \entry
  {2013 - 2017}
  {Bachelor's Degree in Computer Science}
  {Wrocław University of Science and Technology}
  {
    \emph{Faculty:} Fundamental Problems of Technology \\
    \emph{Thesis: ,,Computer modeling and solving geometry puzzles that require collision detection''.} \\
    \emph{Solver of 'Snake cube puzzle'. Application consist of an interactive 3D GUI that allows the user
      to manipulate the model and to solve it step by step. \textbf{C++11 / OpenGL} }
  }
\end{entrylist}


\section{Certifications}
\begin{itemize}
  \setlength\itemsep{-0.32em}
  \item {\small Best practices of object-oriented programming in C++ language}
  \item {\small ISTQB Certified Tester Foundation Level}
  \item {\small E-UTRAN/LTE Signalling}
  \item {\small LTE Cellular IoT}
\end{itemize}


\section{Tools}
\begin{itemize}
  \setlength\itemsep{-0.32em}
  \item \textbf{Cloud/DevOps:} Azure, Docker, Jenkins, Kubernetes
  \item \textbf{Programming:} Bison, Cmake, Cplex, GTest, OpenGL, PyTest
  \item \textbf{Codebase and CI:} Fisheye, Gerrit, Git, Gitlab, Jenkins, Svn
  \item \textbf{Machine Learning:} Keras, Matplotlib, Pandas, Scikit-learn
  \item \textbf{Operating Systems:} Linux, SystemD, Windows
  \item \textbf{Misc:} Jira
\end{itemize}


\section{Skills}
\begin{itemize}
  \setlength\itemsep{-0.32em}
  \item \textbf{Languages:} Polish (Native), English (B2)
  \item \textbf{Programming:} C/C++ (Good), Python (Good), Rust (Intermediate),
            Julia (Intermediate), Prolog (Intermediate), SQL (Basic)
  \item \textbf{Personal:} Team player, Fast learner, Sharing knowledge
\end{itemize}


\section{Interested in}
\begin{itemize}
  \setlength\itemsep{-0.32em}
  \item Dance and sport
  \item Machine learning
  \item Algorithmics
  \item Guitar
\end{itemize}


\vspace{5mm}
{\scriptsize 
  I agree to the processing of personal data provided in this document for realising
  the recruitment process pursuant to the Personal Data Protection Act of 10 May 2018 (Journal of
  Laws 2018, item 1000) and in agreement with Regulation (EU) 2016/679 of the European Parliament
  and of the Council of 27 April 2016 on the protection of natural persons with regard to the
  processing of personal data and on the free movement of such data, and repealing Directive 95/46/EC
  (General Data Protection Regulation).
}
\end{document}
