%!TEX TS-program = xelatex
% \def\print{}
\documentclass[tikz]{friggeri-cv}
\usepackage{afterpage}
\usepackage{hyperref}
\usepackage{color}
\usepackage{xcolor}
\usepackage{graphicx}


\hypersetup{
    pdftitle={},
    pdfauthor={},
    pdfsubject={},
    pdfkeywords={},
    colorlinks=false,       % no link border color
   allbordercolors=white    % white border color for all
}

\newcommand{\circleBoxCoding}{
  \resizebox{.8\linewidth}{!}{
  \begin{tikzpicture}[font=\sffamily\bfseries\large,text=white,]
    \foreach \angle/\col [remember=\angle as \last (initially 0)] in 
        {100/darkgray, 200/pblue, 240/orange, 280/purple, 320/red, 360/green}{
        \draw[\col, line width=55] (\last:2cm) arc[start angle=\last, end angle=\angle, radius=2cm];
        \draw[white, line width=1mm] (\last:0.9)--++(\last:2.3);
    }
    % \node[line width=1mm, draw, circle, minimum width=2.5cm, white, fill=purple!80] {C/C++};
    \node at (50:2cm) {Python};
    \node at (150:2cm) {C++};
    \node at (220:2cm) {Java};
    \node at (260:2cm) {Go};
    \node at (300:2cm) {Julia};
    \node at (340:2cm) {Prolog};
  \end{tikzpicture}
  }
}


\begin{document}
\header{Tomasz}{Kulik}{}
 %Fake text to add separator
\fcolorbox{white}{gray}{\parbox{\dimexpr\textwidth-2\fboxsep-2\fboxrule}{.....}}

% In the aside, each new line forces a line break
\begin{aside}
  \includegraphics[width=0.7\textwidth]{me}
    ~
  \textbf{Birthdate:} 02-06-1994
  \textbf{Phone:} +48 660 718 720
  \href{mailto:}{\textbf{tomek.kulik2@}gmail.com}
  \href{https://linkedin.com/in/tomkulik}{\textbf{linkedin.com}/in/tomkulik}
  \href{https://github.com/kulikthebird}{\textbf{github.com}/kulikthebird}
    ~
  \section{Programming}
    \circleBoxCoding
  \section{Tools}
    ~
    \fbox{Git} \fbox{SVN} \fbox{CMake} \fbox{GMock} \fbox{GTest} \fbox{PyTest} \fbox{Pylint} \fbox{Gitlab} \fbox{Jenkins} \fbox{Jira} \fbox{Fisheye} \fbox{Bison} \fbox{Flex} \fbox{CPLEX} \fbox{Scikit-learn} \fbox{Keras} \fbox{Pandas} \fbox{Matplotlib} \fbox{OpenGL} \fbox{Linux (bash)} \fbox{Windows}
    ~
  \section{Personal Skills}
    Team player
    Fast learner
    Sharing knowledge
    ~
  \section{Languages}
    \textbf{Polish} - Native
    \textbf{English} - Good
    ~
  \section{Interested in}
    Algorithmics
    Machine learning
    Dancing
    Sport
    Guitar
\end{aside}

\section{About me}
\small
{
I have almost 5 years of professional experience in IT, 3 years as a Automation Tester and 2 years as a Software Engineer.
I started as a Working Student in Nokia during a 2nd year of study. I gained a lot of knowledge of Telco industry and
Software project Life Cycle. I had a chance to work on both a very large projects with thousands lines of code
and smaller applications written from scratch. I'm using SOLID principles, well known design patterns and unit testing in my daily basis.
My studies were focused mainly on theory of Computer Science, however there were a lot of practical exercises. I was using a number of
programming languages during labs. I learned a lot of algorithms from different fields such as networking, distributed systems,
coding and compression of data, parsing and compilers' architecture, machine learning, optimization etc.
}

\section{Experience}
\begin{entrylist}
  \entry
    {08/18 - present}
    {Engineer Software Development}
    {Nokia, Wrocław, Poland}
    {
      \vspace{-3mm}
      \begin{itemize}
        \setlength\itemsep{0.2em}
        \item Development of LTE and IoT software in System Module and Radio Module
        \item Implementing new features and maintaining legacy code
        \item Faults analysis
        \item Python3 and C++14 code reviewer, member of a Scrum team
        \item Programming: C++14, Python3, TTCN-3, VBA
      \end{itemize}
    }
  \entry
    {06/15 - 08/18}
    {Software Test Engineer}
    {Nokia, Wrocław, Poland}
    {
      \vspace{-3mm}
      \begin{itemize}
        \setlength\itemsep{0.2em}
        \item Functional Testing, Integration Testing
        \item Development of a test environment for LTE software (Base station)
        \item Creating automated tests, preparing documentation of test plans etc.
        \item Python3 code reviewer, member of a Scrum team
        \item Programming: Python3, RobotFramework
      \end{itemize}
    }
\end{entrylist}

\section{Education}
\begin{entrylist}
  \entry
    {2017 - 2019}
    {Master's Degree in Computer Science (Algorithmics)}
    {Wrocław University of Science and Technology}
    {
      \emph{Faculty:} Fundamental Problems of Technology \\
      \emph{Thesis: ,,Filtering algorithms in constraints programming''.} \\
      \emph{Description and implementation of algorithms for domains reduction
            (constraints propagation) within IBM CPLEX environment. \textbf{C++11 / Optimization} }
    }
  \entry
    {2013 - 2017}
    {Bachelor's Degree in Computer Science}
    {Wrocław University of Science and Technology}
    {
      \emph{Faculty:} Fundamental Problems of Technology \\
      \emph{Thesis: ,,Computer modeling and solving geometry puzzles that require collision detection''.} \\
      \emph{Solver of 'Snake cube puzzle'. Application consist of an interactive 3D GUI that allows the user
            to manipulate the model and to solve it step by step. \textbf{C++11 / OpenGL} }
    }
\end{entrylist}

\section{Certifications}
\begin{itemize}
  \setlength\itemsep{-0.32em}
  \item {\small Best practices of object-oriented programming in C++ language}
  \item {\small ISTQB Certified Tester Foundation Level}
  \item {\small E-UTRAN/LTE Signalling}
  \item {\small LTE Cellular IoT}
\end{itemize}

\vspace{5mm}
{\scriptsize I agree to the processing of personal data provided in this document for realising
the recruitment process pursuant to the Personal Data Protection Act of 10 May 2018 (Journal of
Laws 2018, item 1000) and in agreement with Regulation (EU) 2016/679 of the European Parliament
and of the Council of 27 April 2016 on the protection of natural persons with regard to the
processing of personal data and on the free movement of such data, and repealing Directive 95/46/EC
(General Data Protection Regulation).}

\end{document}
