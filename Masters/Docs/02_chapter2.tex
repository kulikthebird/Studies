\chapter{Analiza problemu}
\thispagestyle{chapterBeginStyle}
Rozdział zawiera opis ograniczenia filtrującego dziedziny przy pomocy automatu skończonego. Omówiono schemat tworzenia
automatu $NFA$ na podstawie wyrażenia regularnego metodą Thompsona. Sposoby opisu globalnych ograniczeń przy pomocy automatów
skończonych z licznikiem zostały wprowadzone w różnych implementacjach języka Prolog, co pozwoliło na usprawnienie procesu modelowania~
\cite{Automata}. Praca skupia się jednak na automatach skończonych, które nie zawierają liczników, co pozwala na wprowadzenie
wygodniejszego interfejsu dla użytkownika (wyrażenia regularne zamiast opis automatu skończonego) oraz większą wydajność.

\section{Założenia}
\par
Algorytm filtrujący ma za zadanie zredukować dziedziny zmiennych celem skrócenia poszukiwania dopuszczalnego rozwiązania.
Dla zadanego wyrażenia regularnego $R$ oraz sekwencji zmiennych $(X_1, X_2, ... , X_n)$ proponowany algorytm
powinien rozpropagować wśród zadanych zmiennych ograniczenie polegające na akceptacji jedynie takich krotek
$(a_1, a_2, ... , a_n)$, które są akceptowane przez automat skończony równoważny z zadanym $R$. Dla przykładu
propagacja w ciągu zmiennych binarnych $(X_1, X_2, X_3, X_4)$ wyrażenia \textit{"0* 1 1 1 0*"} powinna zawęzić dziedziny
zmiennych $X_2$ oraz $X_3$ do wartości $1$. W przypadku stwierdzenia, że dla zmiennych o danych dziedzinach nie jest
możliwe wartościowanie akceptowane przez automat, algorytm powinien zgłosić naruszenie ograniczenia.
\par
W celu zinterpretowania wyrażenia regularnego na potrzeby propagacji konieczne jest stworzenie na jego podstawie
automatu skończonego. W dalszej części rozdziału omówiony został sposób kompilacji wyrażeń regularnych oraz krótka analiza
pozytywnych oraz negatywnych cech wybranych metod. Ostatnia część zawiera znane w literaturze algorytmy
wykorzystujące automaty niedeterministyczne do testowania zadanego tekstu pod kątem akceptacji przez automat.
W kolejnym rozdziale przedstawiono modyfikacje omówionych algorytmów pozwalające na filtrowanie dziedzin zmiennych
w problemach programowania z ograniczeniami.

\section{Kompilacja wyrażenia regularnego}
\par
Istnieje co najmniej kilka sposobów na przetłumaczenie wyrażenia regularnego na automat skończony. W przypadku problemów
wyszukiwania wzorca w tekście jedną z możliwości jest kompilacja do automatu $DFA$, ta jednak wymaga asymptotycznie
wykładniczej ilości pamięci w stosunku do długości wyrażenia regularnego w najgorszym przypadku. Jest to powód, który
czyni tę metodę niepraktyczną pomimo czasu działania zależnego liniowo od długości tekstu do przeszukania.
\par
Drugą metodą jaka się nasuwa jest kompilacja do $NFA$. Przykładem może być algorytm Thompsona~\cite{Thompson}, który na podstawie wyrażenia
regularnego ,,składa'' automat niedeterministyczny przy pomocy prostych reguł zdefiniowanych dla podstawowych
operatorów języków regularnych opisanych w pierwszym rozdziale. Podręcznik "Handbook of Formal Languages vol. 2"~\cite{FormalLanguages2}
wspomina o metodzie Thompsona oraz opisuje drugą wersję pozbawioną tzw. $\epsilon$-przejść pomiędzy stanami. Dla uproszczenia analizy
w pracy tej wykorzystano jednak pierwszą z nich.
\begin{theorem}
Metoda Thompsona ma złożoność czasową i pamięciową kompilacji wyrażenia regularnego do $NFA$ ograniczoną przez $O(|r|)$.
\end{theorem}
\begin{figure}
	\centering
	\begin{center}
	\begin{subfigure}{0.45\textwidth}
		\resizebox{\linewidth}{!}{
			\begin{tikzpicture}[>=stealth',shorten >=1pt,auto,node distance=1.0cm]
				\node[state,initial] (q_0)   {};
				\node[state] (q_1) [above right=of q_0] {};
				\node[state] (q_11) [right=of q_1] {};
				\node[state] (q_2) [below right=of q_0] {}; 
				\node[state] (q_22) [right=of q_2] {};
				\node[state,accepting](q_3) [below right=of q_11] {};
				\node [draw=black!50, fit={(q_1) (q_11)}] {};
				\node [draw=black!50, fit={(q_2) (q_22)}] {};
				\path[->] 
					(q_0) edge  node  {$\epsilon$} (q_1)
						edge  node  {$\epsilon$} (q_2)
					(q_11) edge  node  {$\epsilon$} (q_3)
					(q_22) edge  node  {$\epsilon$} (q_3);
			\end{tikzpicture}
		}
		\caption{Konstrukcja alternatywy.}
	\end{subfigure}
	\end{center}
	\hspace{15pt}
	\begin{center}
	\begin{subfigure}{0.45\textwidth}
		\resizebox{\linewidth}{!}{
			\begin{tikzpicture}[>=stealth',shorten >=1pt,auto,node distance=1.0cm]
				\node[state,initial] (q_0)   {};
				\node[state] (q_1) [right=of q_0] {};
				\node[state] (q_2) [right=of q_1] {}; 
				\node[state,accepting](q_3) [right=of q_2] {};
				\node [draw=black!50, fit={(q_0) (q_1)}] {};
				\node [draw=black!50, fit={(q_2) (q_3)}] {};
				\path[->] 
					(q_1) edge  node  {$\epsilon$} (q_2);
			\end{tikzpicture}
		}
		\caption{Konstrukcja konkatenacji.}
	\end{subfigure}
	\end{center}
	\hspace{15pt}
	\begin{center}
	\begin{subfigure}{0.45\textwidth}
		\resizebox{\linewidth}{!}{
			\begin{tikzpicture}[>=stealth',shorten >=1pt,auto,node distance=1.0cm]
				\node[state,initial] (q_0)   {}; 
				\node[state] (q_1) [right=of q_0] {};
				\node[state] (q_2) [right=of q_1] {}; 
				\node[state,accepting](q_3) [right=of q_2] {};
				\node [draw=black!50, fit={(q_1) (q_2)}] {};
				\path[->] 
					(q_0) edge [bend right=35] node  {$\epsilon$} (q_3)
						edge node  {$\epsilon$} (q_1)
					(q_2) edge [bend right=35]  node  {$\epsilon$} (q_1)
						edge node  {$\epsilon$} (q_3);
			\end{tikzpicture}
		}
		\caption{Konstrukcja operatora gwiazdki - domknięcia Kleene'ego.}
	\end{subfigure}
	\end{center}
	\caption{Powyższa figura przedstawia struktury potrzebne do skonstruowania automatu $NFA$ metodą Thompsona~\cite{Thompson}.}
\end{figure}

\section{Działanie automatu}
\par
Choć sama kompilacja działa w czasie liniowym oraz wymaga wielkości pamięci liniowo zależnej od długości wyrażenia regularnego, to
kluczowym problemem jest tutaj sam algorytm przetwarzający tekst wejściowy zgodnie z automatem. Metoda przeszukiwania z nawrotami
wymaga wykładniczego czasu względem liczby stanów automatu, ponieważ każde miejsce, w którym występuje niedeterministyczne przejście
pomiędzy stanami jest przeszukiwane oddzielnie.
\par
Istnieje jednak algorytm, który ,,symuluje'' niedeterminizm występujący w automatach $NFA$ - \textit{RegularExpressionTester}~
\cite{FormalLanguages2}. Zasada jego działania polega na tym, aby utrzymywać zbiór stanów, w którym aktualnie
znajduje się automat. Innymi słowy zamiast jednego stanu w danym czasie automat może znajdować się w wielu stanach,
aktualizując każdy ze stanów w czasie wczytywania kolejnych znaków z wejścia. Wartość zwrócona przez funkcję
\textit{RegularExpressionTester} mówi o tym, czy wejściowy ciąg znaków został zaakceptowany przez automat.
\begin{theorem}
Metoda \textit{RegularExpressionTester} z podręcznika ,,Handbook of Formal Languages''~\cite{FormalLanguages2} ma złożoność czasową
zależną od długości wejścia i liczby stanów $O(n*k)$. Złożoność pamięciowa wynosi $O(k)$.
\end{theorem}
\begin{figure}
	\centering
	{\small
		\begin{pseudokod}[H]
		\SetArgSty{normalfont}
		\SetKwProg{Fn}{Function}{}{}
		\SetKwFunction{Enqueue}{Enque}
		\SetKwFunction{Dequeue}{Dequeue}
		\Fn{Closure($E$, $S$)}
		{
			R $\leftarrow$ S \\
			Q $\leftarrow$ empty queue \\
			\ForEach{$p \in S$}
			{
				\Enqueue{Q, $p$}
			}
			\While{Q is not empty}
			{
				p $\leftarrow$ \Dequeue{Q} \\
				\ForEach{state $q$ such that $(p, \epsilon, q) \in E$}
				{
					\If{$q \notin R$}
					{
						R $\leftarrow$ $R + {q}$ \\
						\Enqueue{Q, q}
					}
				}
			}
			\Return{R}
		}
		\caption{Algorytm znajdujący domknięcie danego stanu, czyli zbiór wszystkich stanów osiągalnych przez $\epsilon$-ścieżki
				od stanu wejściowego $S$.}
		\label{alg:Closure}
		\end{pseudokod}
	}
\end{figure}

\begin{figure}
	\centering
	{\small
		\begin{pseudokod}[H]
		\SetArgSty{normalfont}
		\SetKwProg{Fn}{Function}{}{}
		\Fn{Transitions($E, S, a$)}
		{
			R $\leftarrow$ empty set \\
			\ForEach{$p \in S$}
			{
				\ForEach{state $q$ such that $(p, a, q) \in E$}
				{
					R $\leftarrow$ $R + {q}$
				}
			}
			\Return{R}
		}
		\caption{Algorytm symulujący niedeterminizm automatu $NFA$ w czasie wielomianowym i pamięci ograniczonej wielomianem
				od liczby stanów.}
		\label{alg:Transitions}
		\end{pseudokod}
	}
\end{figure}

\begin{figure}
	\centering
	{\small
		\begin{pseudokod}[H]
		\SetArgSty{normalfont}
		\SetKwProg{Fn}{Function}{}{}
		\SetKwFunction{Enqueue}{Enque}
		\SetKwFunction{Dequeue}{Dequeue}
		\SetKwFunction{Closure}{Closure}
		\SetKwFunction{Transitions}{Transitions}
		\Fn{RegularExpressionTester($x$, $y$)}
		{
			$(Q, i, {t}, E)$ $\leftarrow$ $NFA$ \\
			$C$ $\leftarrow$ \Closure{$E, {i}$} \\
			\ForEach{letter $a$ from input text}
			{
				$C$ $\leftarrow$ \Closure{$E$, \Transitions{$E, C, a$}} \\
			}
			\Return{$t \in C$}
		}
		\caption{Algorytm przetwarzający tekst wejściowy zgodnie ze skonstruowanym wcześniej automatem $NFA$. Wyjściem algorytmu 
		jest wartość $TRUE$ jeśli tekst jest akceptowany przez automat lub $FALSE$ w przeciwnym przypadku.}
		\label{alg:RegularExpressionTester}
		\end{pseudokod}
	}
\end{figure}
