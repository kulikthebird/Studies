\chapter{Implementacja}
\thispagestyle{chapterBeginStyle}
Rozdział jest opisem wykonanej implementacji do ograniczenia $RegexConstraint$. Zawarte zostały ważniejsze części kodu kompilującego
wyrażenie regularne oraz algorytmu analizującego sekwencję zmiennych pod kątem wygenerowanego automatu. Implementacja została
wykonana w języku C++11 z użyciem algorytmów zawartych w bibliotece standardowej oraz pakietu IBM ILOG CPLEX CP Optimizer dla języka
C++. Równoważną implementację można wykonać dla języka Java (ze względu na wsparcie ze strony pakietu), jednak wykracza
to poza zakres tej pracy.


\section{Struktury danych}
\par
Ważną strukturą, która reprezentuje przejście pomiędzy dwoma stanami jest \textit{Transition} (patrz kod \ref{alg:Structures}).
Przechowuje ona wartość akceptowaną przez daną relację oraz stan, do którego automat powinien przejść. Dodatkową
informacją jest pole \textit{isEmptyString}, które mówi o tym, czy dane przejście ma być tzw. $\epsilon - transition$,
czyli przejściem bezwarunkowym. Takie przejście nie pobiera kolejnego znaku z ciągu wejściowego, zamiast tego
odbywa się pobierając tzw. pusty łańcuch znaków. Intuicyjnie jeśli automat znajduje się w stanie, w którym są
$\epsilon - transition$ to może on niedeterministycznie przełączyć w stany, na które te przejścia pozwalają.
\par
W zaproponowanej implementacji strukturą reprezentującą sam stan jest wektor przejść stanowych. Kolejne indeksy
tego wektora oznaczają kolejne możliwe wyjścia z tego stanu.

\begin{figure}
\begin{lstlisting}[caption=Struktury wykorzystywane w czasie konstruowania automatu $NFA$ oraz propagacji ograniczenia.,
    style=customc, label=alg:Structures]
/*
 * Structures used by RegexConstraint
 */
typedef unsigned int StateNumber;
struct Transition
{
    int value;
    StateNumber newState;
    bool isEmptyString;
};
typedef std::vector<Transition> State;
typedef std::vector<State> StatesVector;
\end{lstlisting}
\end{figure}


\section{Główna struktura ograniczenia}
\par
Zaimplementowana klasa \textit{RegexConstraintI} jest strukturą przechowującą wektor wygenerowanych stanów,
wektor zmiennych z sekwencji wejściowej oraz wektor wartości dopuszczalnych umieszczonego tu ze względu
na optymalizację (raz przygotowana struktura, aby algorytm nie musiał alokować jej za każdym krokiem propagacji).
Wartym uwagi konstruktorem jest ten, który przyjmuje jako argumenty wyrażenie regularne oraz wektor zmiennych. W czasie jego
działania następuje konstrukcja automatu $NFA$, który będzie później wielokrotnie używany w procesie filtrowania dziedzin.
Działanie głównych funkcji/metod zawartych w tej klasie zostało opisane w rozdziale 3.

\begin{figure}
\begin{lstlisting}[caption=Główna klasa zawierająca implementację ograniczenia \textit{RegexConstraint}., style=customc]
/**
*  Class used by IBM CP solver's engine.
*/
class RegexConstraintI : public IloPropagatorI
{
public:

    /*
    * Constructor of regex constraint. It compiles regex to NFA and prepares feasibleValues vector.
    */
    RegexConstraintI(std::vector<IloIntVar> vars, std::string regex);

    /*
    * Constructor used by IBM CP engine.
    */
    RegexConstraintI(StatesVector states,
                    std::vector<StatesVector> rect,
                    std::vector<IloIntVar> vars);

    /*
    * Function used by IBM CP engine
    */
    IloPropagatorI* makeClone(IloEnv env) const override;

    /*
    * Method that is called during constraint propagation. The result of this method's work are reduced domains.
    */
    void execute() override;

private:
    void resetFeasibleValuesVectors();
    void addState(State&& state);
    int parseRegex(const std::string& str, uint index);
    std::set<StateNumber> evaluateAutomatonWithInputVars();
    std::set<StateNumber> findFinishStatesAfterSearch(std::set<StateNumber>& statesAfterSearching);
    void reduceVarsDomains(std::set<StateNumber>& finishStates);

    /*
    * A vector used to store states created during regex compilation.
    */
    StatesVector states;

    /*
    * A vector of feasible values filled during automaton evaluation. For every state and every input variable
    * the algorithm allocates a vector of states and values from wich given state was reached.
    */
    std::vector<StatesVector> feasibleValues;

    /*
    * A vector of IBM CP integer variables.
    */
    std::vector<IloIntVar> vars;
};
\end{lstlisting}
\end{figure}


\section{Parsowanie wyrażeń regularnych}
\par
Tłumaczenie wyrażeń regularnych na automaty skończone służy do wstępnego przygotowania danych dla algorytmu filtrującego.
W literaturze opisano co najmniej kilka metod, m.in. metodę Thompsona. W zaproponowanej implementacji użyta została
wspomniana metoda (jej interpretacja). Istnieją metody, które lepiej dysponują zasobami w postaci czasu działania oraz pamięci.
Pozwala to na dalszy rozwój oraz optymalizację proponowanego algorytmu.
\par
Funkcja \textit{parseRegex} (patrz kod \ref{alg:parseRegex}) służy do tłumaczenia wyrażeń regularnych na automaty skończone $NFA$. Jest to interpretacja
metody Thompsona omówionej w poprzednich rozdziałach. Ze względu na bezkontekstowość gramatyki wyrażeń regularnych spowodowanej
dodaniem zagnieżdżonych nawiasów należało wprowadzić rekurencyjne odwołania w przypadku napotkaniu symbolu ,,$($''. W celu
umożliwienia wprowadzania liczb większych niż jednocyfrowe, gramatyka zakłada, że kolejne liczby powinny być oddzielone od
siebie znakiem innym niż cyfra. Dla przykładu w przypadku, gdy użytkownik dla zmiennych $X_1, X_2, X_3$ zechce narzucić ograniczenie,
aby $X_1 = 10$, $X_2 = (22)^*$ i $X_3 = 30$, wystarczy określić wyrażenie: $10 \; 22^*30$.
\par
Wspomniana funkcja iteruje się po kolejnych symbolach z zadanego łańcucha znaków, zaczynając od danego indeksu (jako pierwsze wywołanie
metoda zaczyna analizę od początku ciągu, czyli indeksu równemu $0$). Główną częścią tej metody jest instrukcja \textit{switch},
która w przypadku napotkania danego symbolu odpowiednio go interpretuje dodając nowy stan, bądź nowe przejścia do istniejących stanów.
W przypadku wystąpienia nawiasów funkcja odpowiednio wykonuje rekurencyjnie samą siebie, gdy napotka nawias otwierający oraz powraca
do poprzedniego wywołania napotykając nawias zamykający. Za każdym takim razem funkcja dodaje do automatu stan oznaczający przejście po wszystkich
symbolach zawartych wewnątrz nawiasu oraz zwraca ostatni przetwarzany indeks, aby funkcja, która ją wywołała mogła kontynuować przetwarzanie
od tego miejsca.
\par
Gdy funkcja napotka znak gwiazdki $*$ (domknięcie Kleene'ego) to stworzy w automacie pętlę przez dodanie $\epsilon$-przejścia do początku
ostatniego bloku. Znak $?$ powoduje akceptację poprzedniego symbolu/bloku lub bezwarunkowe pominięcie tego symbolu. W przypadku,
gdy analizowanym znakiem będzie $|$ (alternatywa), algorytm zachowa się różnie w zależności od tego, czy kolejnym elementem będzie
symbol czy blok zagnieżdżony pomiędzy nawiasami.


\begin{figure}
\begin{lstlisting}[caption=Metoda kompilująca wyrażenie regularne do automatu NFA, style=customc, label=alg:parseRegex]
int RegexConstraintI::parseRegex(const std::string& str, uint index)
{
    uint prelastState = states.size()-1;
    int currentNumber = 0;
    for(uint i=index; i<str.length(); i++)
        switch(str[i])
        {
            case '(':
                prelastState = states.size();
                i = parseRegex(str, i+1);
            break;
            case ')':
                addState({Transition {-1, (StateNumber) states.size()+1, true}});
                return i;
            break;
            case '*':
                states[states.size()-1][0].newState = prelastState;
                states[prelastState].push_back(Transition {-1, (StateNumber) states.size(), true});
            break;
            case '?':
                states[prelastState].push_back(Transition {-1, (StateNumber) states.size(), true});
            break;
            case '|':
                while(i<str.length()-1 and str[i+1] == ' ')
                    i++;
                if(str[i+1] != '(')
                {
                    states[prelastState].push_back(Transition {-1, (StateNumber) states.size()+1, true});
                    addState({Transition {-1, (StateNumber) states.size()+2, true}});
                }
                else
                {
                    states[prelastState].push_back(Transition {-1, (StateNumber) states.size()+1, true});
                    addState({});
                    prelastState = states.size()-1;
                    i = parseRegex(str, i+2);
                    states[prelastState].push_back(Transition {-1, (StateNumber) states.size(), true});
                }
            break;
            default:
                if(str[i] >= '0' and str[i] <= '9')
                {
                    currentNumber = currentNumber*10 + (str[i] - '0');
                    if(i>=str.length()-1 or str[i+1] < '0' or str[i+1] > '9' )
                    {
                        prelastState = states.size();
                        addState({ Transition {currentNumber, (StateNumber) states.size()+1, false} });
                        currentNumber = 0;
                    }
                }
        }
    return str.length();
}
\end{lstlisting}
\end{figure}
